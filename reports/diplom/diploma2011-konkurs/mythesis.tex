\Title%
{Параллельная реализация метода поиска закономерностей в последовательностях событий}%
{Вишневский Валерий Викторович}%
{Вишневский В.\,В.}%
{valera.vishnevskiy@yandex.ru}%
{Кафедра математических методов прогнозирования}%
{Научный руководитель:  к.ф-м.н., н.с. Ветров Дмитрий Петрович}

В данной дипломной работе предлагается оригинальный алгоритм для поиска 
скрытых закономерностей(паттернов) в последовательностях событий,
основанный на вероятностном представлении паттернов в дискретных 
последовательностях событий, рассматривается применение данного 
алгоритма для анализа поведения мышей. 

Задача поиска закономерностей в поведении животных и людей 
крайне важна в современной нейробиологии и когнитивных науках. 
Выделив характерные паттерны, можно, например, делать выводы
о сложности поведения различных особей, определять изменения в поведении 
наблюдаемых процессов, другими словами, решив задачу поиска паттернов, 
можно определенным образом {\em измерять} поведение.
Анализ поведения является основным инструментом при исследовании 
механизмов работы памяти и обучения на системном уровне .

Используемый на сегодняшний день метод поиска Т"=паттернов, предложенный 
М.\,С.~Магнуссоном в [1] имеет два ключевых недостатка: сильная чувствителен к шуму;
полученные Т"=Паттерны специфичны особи, в поведении которой они были найдены, т.\,е., на 
практике, невозможно определять степень <<похожести>> поведения разных особей.

Разработанный в данной дипломной работе метод, с одной стороны, 
основывается на зарекомендовавшей себя и верифицированной идее Магнуссона: 
поиск паттернов производится снизу вверх, сначала 
находятся простые закономерности, потом, путем их соединения, образуются 
более сложные паттерны. 
В виду введенной вероятностной модели P"=паттерна(probabilistic pattern) 
были полностью пересмотрены этапы конструирования, редукции, и поиска паттернов.
В рамках данной научной работы были получены оценки на уровень <<четкости>>
паттернов, разработан статистический критерий для определения информативных
закономерностей(боле подробно см.~[2,\;3]).

В [2,\;3] показано, что разработанный подход расширяет метод поиска Т"=паттернов:
гарантировано нахождение, как минимум, тех же закономерностей, которые нашел 
метод Магнуссона. 
Более того, показано, что предложенный метод может обрабатывать шум и пропуски
в исходных данных, и чаще находит более длинные и значимые закономерности. 

Метод поиска P"=паттернов имеет 
cложность $O\left(n^3\right)$(алгоритм поиска Т"=паттернов~--- $O\left(n^2\right)$) от числа событий $n$.
% Для возможности применять метод на практике, 
Была разработана параллельная 
версия алгоритма для GPU, работающая примерно
в 40 раз быстрее последовательной версии.
В итоге, характерная 12-ти минутная сегментация поведения лабораторной мыши
обрабатывается примерно 11 секунд.

В заключении работы был проведен анализ работы алгоритма на 
реальных данных поведения мышей. 
Имелась разметка поведения двух групп животных: здоровая контрольная группа, 
и группа особей с поврежденной функцией гиппокампа. 
Были решены следующие задачи.
\begin{itemize}
    \item 
      \emph{Поиск закономерностей.} Эксперты подтвердили, что 
      найденные P"=паттерны имеют четкую, нетривиальную структуру и 
      полностью согласуются с современными представлениями о поведении.
    \item
      \emph{Классификация особей на группы по поведению.}
      Для P"=паттернов получено качество классификации порядка 90--95\% правильных ответов, 
      для Т-паттернов примерно 65--70\%.
    \item
      \emph{Поиск характеристических определенной группе животных паттернов.} При этом оказалось,
      что в <<гиппокампальной>> группе была особь, имеющая много паттернов характерных контрольной группе. 
      Данный факт согласовался с экспериментом: после гистологического
      анализа выяснилось, что гиппокамп одной из мышей не удалось полностью парализовать.
\end{itemize}

Таким образом, разработанный в рамках дипломной работы 
% метод является не только
% инструментом для поиска закономерностей, которые должны быть затем тщательно
% проанализированы экспертом; 
метод поиска P"=паттернов, благодаря вариабельности
найденных закономерностей между разными особями, позволяет описывать 
поведение конкретного животного, вектором <<откликов>> на найденные паттерны,
после чего, могут быть использованы стандартные алгоритмы машинного обучения 
для решения, например, задач классификации, кластеризации, или восстановления регрессии. 
Подчеркнем, что данный подход не имеет аналогов в области анализа поведения. 
Похожий подход \emph{мешка слов(bag-of-words)} активно и успешно 
используется в области компьютерного зрения, что дает повод рассчитывать на успешность
схожих идей в области анализа поведения.

Одна из нетривиальных задач, для решения которой можно использовать предложенный метод~---
оценка <<сложности>> поведения и проверка гипотезы соответствия сложности 
поведения филогенетическому дереву. 
% Проблема анализа сложности поведения
% в рамках эволюции не решена на сегодняшний день, именно из-за того, что было 
% невозможно оценивать в единой шкале поведение разных видов животных.

% В данной работе алгоритм поиска закономерностей рассматривался в контексте применения его для анализа поведения.
% Но, очевидно, заменив понятие поведенческого акта на какое-то абстрактное событие(маркер), мы можем 
% искать закономерности в различных потоках данных. Например, событиями могут быть:
% повышения и понижения курсов валют в анализе поведения рынка;
% аминокислоты, кодоны, или азотистые основания при анализе структуры ДНК;
% всплеск активности отдельного нейрона при анализе спайковой активности нейронных культур;
% новостные тренды(в виде ключевых слов) при анализе закономерностей в политике и обществе.
% Например, алгоритм Т-Паттернов и ранее был использован для анализа структуры 
% ДНК~[5] и стратегии футбольных команд во время матчев.


\References
\begin{enumerate}
\item
    M.S.~Magnusson. \emph{Discovering hidden time patterns in behavior:
T-patterns and their detection.}~--- Behavior Research Methods, Instruments, Computers
2000.

\item
    V.V.~Vishnevskiy, D.P~Vetrov. \emph{The Algorithm for Detection of Fuzzy Behavioral Patterns.}
~--- Proceedings of Measuring Behavior 2010, ISBN 978-90-74821-86-5.

\item
    В.В.~Вишневский. \emph{Параллельная реализация метода поиска закономерностей в
последовательностях событий.}~--- Дипломная работа. Факультет ВМК МГУ имени М.\,В.~Ломоносова, 2011.

% \item
%   P.~Martin, P.~Bateson. \emph{Measuring Behaviour: An Introductory Guide.}~--- Cambridge University Press, second edition, 1993.
% 
% \item
% M.S.~Magnusson. \emph{Analyzing complex real-time streams of behavior:
% repeated patterns in behavior and DNA.}~--- Ethologie humaine. Levallois-Perret, France, 2003.

\end{enumerate}
